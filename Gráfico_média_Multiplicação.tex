\documentclass[a4paper,2pt]{report}
\usepackage[utf8]{inputenc}
\usepackage[brazil]{babel}
\usepackage[lmargin=3cm,tmargin=3cm,rmargin=2cm,bmargin=2cm]{geometry}
\usepackage{pgfplots}
\pgfplotsset{compat=1.17}

\begin{document}

\title{RELATÓRIO - GRÁFICO - PROCESSAMENTO DA MULTIPLICAÇÃO - C++}
\author{Teodoro, Lucas}
\date{\today}

\maketitle

\section{Especificações da máquina testada}
\begin{enumerate}
    \item Descrição: Laptop;\\
    \vspace{3pt}
    \item Produto: Inspiron 15-3567 (078B);\\
    \vspace{3pt}
    \item Fabricante: Dell Inc;\\
    \vspace{3pt}
    \item Serial: 9H2X9X2;\\
    \vspace{3pt}
    \item Processador: Intel® Core™ i3-7020U x 4;\\
    \vspace{3pt}
    \item Gráficos: Intel® HD Graphics 620 (KBL GT2F);\\
    \vspace{3pt}
    \item Capacidade de Disco: 1,0 TB;\\
    \vspace{3pt}
    \item Sistema Operacional: Ubuntu 23.04;\\
    \vspace{3pt}
    \item Tipo do SO: 64 bits;
\end{enumerate}

\begin{tikzpicture}
\begin{axis}[
    title={Gráfico de Variação em Nanossegundos},
    xlabel={Teste},
    ylabel={Nível Bateria - Variação (ns)},
    legend pos=east west,
    grid=both,
    grid style={dotted},
    width=1.0\textwidth,
    height=1.5\textwidth,
    xtick=data,
    ymin=0,
]
\addplot[mark=none, color=blue] coordinates {
    (1, 205)
    (2, 240)
    (3, 234.5)
    (4, 240.9)
    (5, 85.5)
    (6, 75.3)
    (7, 72.4)
    (8, 75.9)
    (9, 76.7)
    (10, 56.1)
};
\addlegendentry{Variação (ns) 0-250}

\addplot[mark=none, color=green] coordinates {
    (1, 4)
    (2, 4)
    (3, 4)
    (4, 2)
    (5, 3)
    (6, 98)
    (7, 98)
    (8, 99)
    (9, 99)
    (10, 100)
};
\addlegendentry{Nível da Bateria 0-100}

\end{axis}
\end{tikzpicture}
\end{document}
